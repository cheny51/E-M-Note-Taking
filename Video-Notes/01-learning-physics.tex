\documentclass[12pt, letterpaper]{article}

\usepackage{fullpage}
\usepackage{amsmath}
\usepackage{outlines}

\begin{document}
    \section{Physics: What the heck am I learning?}

    \begin{equation}
        \nonumber
        \begin{split}
            % The math columns
            \underline{\text{Math}} \\
            \int_{}^{}\frac{df}{dx} \,\mathrm{d}x = f
        \end{split}
        \quad \vline \quad
        \begin{split}
            % The Qualitative columns
            \underline{\text{Qualitative}} \\
            \text{Position change over time linearly with no acceleration}         
        \end{split}
        \quad \vline \quad
        \begin{split}
            % The Quantitative columns
            \underline{\text{Quantitative}} \\
            \sum_{}^{}\vec{F}=m\vec{a} \\
            \vec{a} = \frac{d\vec{v}}{df} \\
            \vec{x} = \frac{d\vec{v}}{df} \\
        \end{split}
    \end{equation}

    \section{4 Fields of Physics}
    There 4 kinds of physics:
    \begin{center}
        \begin{tabular}{| c | c |}
          \hline
            classical mechanics & quantum mechanics \\ \hline
            relativity          & QFT \\
          \hline
        \end{tabular}
      \end{center}
    All related to E\&M.

    \section{The 4 Basic Forces}
    \begin{outline}[enumerate]
    \1 Strong force: 
      \2 Bring neutron, proton, electron away, together
    \1 Electromagnectic force:
      \2 Drive electric motor, magnets, give us friction, air resistance, normal forces
    \1 Weak:
      \2 Nuclear decay. beta and gamma.
    \1 Gravity
      \2 Weaker than all of them. Behave like E\&M so use almost the same euqation
    \end{outline}

    \section{Scalars and Vectors}
    \begin{outline}[enumerate]
      \1 Scalars 
        \2 unit-less
        \2 magnitude
        \2 mass
        \2 time
        \2 count
      \1 Vectors
        \2 Length(magnitude) and direction
        \2 Know the representation of a Vectors
        \2 Addition and multiplication is the typical operator
        \2 You can add Vectors, scalar product
          \3 dot product and cross product does scalar computation
        \2 Displacement
      \1 vector addition/subtraction
      \1 dot product and cross product to manipulate vectors
    \end{outline}

    \section{Dot Product and Cross Product}
    \begin{equation}
      \nonumber
      \begin{split}
          % The dot product columns
          \underline{\text{Dot Product}} \\
          \vec{a}\cdot \vec{b} &= c \quad \text{(scalar)} \\
          &= ab \cos{\theta} \quad \\
          \vec{a} \cdot \vec{a} = a^2 \\
          order of operation don't matter
      \end{split}
      \quad \vline \quad
      \begin{split}
          % The cross product columns
          \underline{\text{Cross Product}} \\
          \vec{a} \times \vec{b} &= \vec{c} \quad \text{(vector)} \\
          &= ab \sin{\theta} \hat{n} \\
          \hat{n} \perp \vec{a} \\
          \hat{n} \perp \vec{b} \\
          order of operation matter.\\
          \vec{a} \times \vec{a} = 0
          \quad
      \end{split}
  \end{equation}
  \begin{outline}[enumerate]
  \1 If $a$ double in size, then dot product double in size. 
  \1 If you $a$ double in size, then cross product double in size.
  \1 For cosine term, 0 degree give 1, 90 degree give 0, 180 degree give -1, and so on.
    \2 Therefore, the dot product is maximized when the 2 vector point to the same direction (or how parallel they are)
  \1 For sine term, 0 degree give 0, 90 degree give 1, 180 degree give 0.
    \2 Therefore, the cross product is maximized when the 2 vectos are perpendiculr to each other
  \end{outline}

  \subsection{The Dirac Delta Function}

  \subparagraph{The divergence of $\frac{\dot{r}}{r^2}$}
  \begin{equation}
    \Delta\cdot \vec{v} = \frac{1}{r^2}\frac{\partial}{\partial r}(r^2*\frac{1}{r^2}) = 0 \quad \text{The divergence is 0} \\
    
    \int_{volume}^{}\Delta \cdot \vec{v}  \,\mathrm{x} = \oint_{surface}{}v   \, \mathrm{\vec{a}} \\


    Gauss's theorem
  \end{equation}

\end{document}